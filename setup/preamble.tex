
% used pagages
\usepackage     [utf8]                  {inputenc}
\usepackage     [T1]                    {fontenc}
\usepackage                             {color}
\usepackage                             {amsmath}
\usepackage                             {graphicx}
\usepackage     [english]               {babel}
\usepackage                             {hyperref}


% links
\definecolor{darkblue}{rgb}{0.0,0.0,0.4}
\definecolor{darkgreen}{rgb}{0.0,0.4,0.0}
\hypersetup{
    colorlinks=true,
    linktoc=all,
    linkcolor=black,
    citecolor=darkgreen,
    urlcolor=darkblue
}

%%%%%%%%%%%%%%%%%%%%%%%%%%%%%%%%%%%%%%%%%%%%%%%%%%%%%%%%%%%%%%%%%%%%%%%%
%%% more configuration for KomaScript %%%

%% the KOMA-Script classes will typeset sectioning headings in sans-serif
%% switch to serif for chapter headings
%\setkomafont{disposition}{\bfseries}

%% set both chapter name and page number to the header
\usepackage[automark]{scrlayer-scrpage}
%\addtokomafont{pagenumber}{\oldstylenums}
\clearpairofpagestyles
\lehead{\leftmark}
\rohead{\rightmark}
\renewcommand*\chaptermarkformat{\thechapter\autodot\enskip} % remove the word "Chapter" in the header
%\ohead{\pagemark}
\ofoot*{\pagemark} % set the page number at the footer, add * to apply for both scrheadings and plain pagestyle
\pagestyle{scrheadings}

%% for sub figure and sub figure caption
\usepackage{caption}
\usepackage{subcaption}
%\setkomafont{caption}{\itshape\sffamily}
\setkomafont{captionlabel}{\bfseries}

%% indent of multi-line caption with KomaScript
\setcapindent{0pt}

%% for setting the vertical space in the labeling environment in the KOMA-script
\usepackage{xpatch}
\setkomafont{labelinglabel}{\bfseries}%{\ttfamily}
\setkomafont{labelingseparator}{\normalfont}
\xpatchcmd{\labeling}
  {\let\makelabel\labelinglabel}
  {\let\makelabel\labelinglabel\itemsep-4pt}
  {}
  {}

%%%%%%%%%%%%%%%%%%%%%%%%%%%%%%%%%%%%%%%%%%%%%%%%%%%%%%%%%%%%%%%%%%%%%%%%
%%% bibliography %%%

%% use biblatex for bibliography
\usepackage{csquotes}
\usepackage[style=numeric-comp, sorting=none, maxnames=4]{biblatex}

%%% the bib file is added in the preamle for biblatex
\addbibresource{references.bib}

%%%%%%%%%%%%%%%%%%%%%%%%%%%%%%%%%%%%%%%%%%%%%%%%%%%%%%%%%%%%%%%%%%%%%%%%
%% citation

%% for a capitalized autoref name
\addto\extrasenglish{%
  \def\chapterautorefname{Chapter}%
}

%%%%%%%%%%%%%%%%%%%%%%%%%%%%%%%%%%%%%%%%%%%%%%%%%%%%%%%%%%%%%%%%%%%%%%%%
%%% customize fonts %%%

%%% If missing some symbols
\usepackage{amsthm}
\usepackage[psamsfonts]{amssymb, amsfonts}
%%%%%%%%%%%%%%%%%%%%%%%%%%%%%%%%%%%%
%%% Latin Modern Roman, enhanced "default latex fonts"
% \usepackage{lmodern}
% \usepackage[scaled=1.05]{zlmtt}% latin modern typewriter

%%% DejaVu sans & typewriter
%\usepackage[scaled=0.94]{DejaVuSansMono}%typewriter
%\usepackage[scaled=0.94]{DejaVuSans}% sans serif

%%%%%%%%%%%%%%%%%%%%%%%%%%%%%%%%%%%%
%%% libertine fonts
% \usepackage{textcomp}
% \usepackage[sb]{libertine}
% \usepackage[scaled=1.04,varqu,varl]{inconsolata}% inconsolata
% \usepackage[libertine,bigdelims,vvarbb]{newtxmath} % bb from STIX
% \usepackage[cal=pxtx]{mathalfa} % mathcal
% \usepackage[supstfm=libertinesups, supscaled=1.2, raised=-.13em]{superiors}

%%%%%%%%%%%%%%%%%%%%%%%%%%%%%%%%%%%%
%%% newtx family fonts (Times like fonts)
\usepackage[full]{textcomp}
\usepackage{newtxtext}
\usepackage[scaled=1.05,varqu,varl]{inconsolata}% sans serif typewriter
% \usepackage[zerostyle=c]{newtxtt} % serifed typewriter
\usepackage[bigdelims,vvarbb]{newtxmath} % bb from STIX
%\usepackage[cal=pxtx]{mathalfa} % mathcal

%%%%%%%%%%%%%%%%%%%%%%%%%%%%%%%%%%%%
%%% xcharter fonts
% \usepackage[full]{textcomp}
% \usepackage[sups,scaled=.98]{XCharter}
% \usepackage[scaled=0.90]{DejaVuSans}% sans serif
% %\usepackage[scaled=0.90]{DejaVuSansMono}% another option for typewrite
% \usepackage[scaled=1.05,varqu,varl]{inconsolata}% typewriter
% \usepackage[charter,bigdelims,vvarbb,scaled=1.07]{newtxmath}% math
% %\usepackage[cal=pxtx]{mathalfa} % mathcal
% \linespread{1.04}

%%%%%%%%%%%%%%%%%%%%%%%%%%%%%%%%%%%%
%\usepackage{bm} % load after all math to give access to bold math

%%%%%%%%%%%%%%%%%%%%%%%%%%%%%%%%%%%%%%%%%%%%%%%%%%%%%%%%%%%%%%%%%%%%%%%%
%%% other configurations %%%

%% dummy text and page layout debugging
\usepackage{blindtext}
%\usepackage{showframe}

%% line number
\usepackage{lineno}

%% set line spacing
\usepackage{setspace}
\onehalfspacing

%% standalone
\usepackage{standalone}
%% tikz for figure annotation
\usepackage{tikz}
\usetikzlibrary{shapes,arrows}
\usetikzlibrary{decorations.pathreplacing}
\usetikzlibrary{circuits.logic.US}

%% use the tikz-timing package to draw timing diagram
\usepackage{tikz-timing}

%% draw Feynman diagrams
\usepackage[force]{feynmp-auto}

%% for tables
\usepackage{array}
\usepackage{booktabs}
\usepackage{tabu}
\usepackage{multirow}

%% units
\usepackage{siunitx}
\sisetup{
	range-phrase = -,
	exponent-product = \cdot,
	range-units = single,
	per-mode = symbol,
	detect-all,
	math-micro=\upmu,text-micro=\ensuremath{\upmu},
	math-ohm=\upOmega,text-ohm=\ensuremath{\upOmega},
}

%% xspace for space after newcommand
\usepackage{xspace}

%% separation of list items
\usepackage{enumitem}

%% reset footnote number per page
\usepackage{perpage}
\MakePerPage{footnote}

%% for showing source code
\usepackage{listings}
\lstset{
	numbers=left,
	numbersep=5pt,
	numberstyle=\tiny\color[gray]{0.6},
	frame=tb,
	aboveskip=3mm,
	belowskip=3mm,
	keepspaces=true,
	showstringspaces=false,
	basicstyle=\small\ttfamily,
	keywordstyle=\color{blue},
	commentstyle=\color[grey]{0.6},
	stringstyle=\color[RGB]{255,150,75}
}
\definecolor{lightgray}{gray}{0.9}
\newcommand{\inlinecode}[2]{\colorbox{lightgray}{\lstinline[language=#1]$#2$}}
